\documentclass[10pt]{article}
\usepackage[usenames]{color} %used for font color
\usepackage{amssymb} %maths
\usepackage{amsmath} %maths
\usepackage[utf8]{inputenc} %useful to type directly diacritic characters
\begin{document}
\[\documentclass[10pt]{article}
 
\usepackage[margin=1.5cm]{geometry} 
\usepackage{amsmath,amsthm,amssymb}
\usepackage{polski}
\usepackage[utf8]{inputenc}
\usepackage{siunitx}
\usepackage{graphicx}
\usepackage{comment}
\usepackage[font=scriptsize]{caption}
\usepackage{subcaption} 
\usepackage{array}


\newcolumntype{C}[1]{>{\centering\let\newline\\\arraybackslash\hspace{0pt}}m{#1}}


 
\newenvironment{theorem}[2][Twierdzenie]{\begin{trivlist}
\item[\hskip \labelsep {\bfseries #1}\hskip \labelsep {\bfseries #2.}]}{\end{trivlist}}
\newenvironment{question}[2][Pytanie]{\begin{trivlist}
\item[\hskip \labelsep {\bfseries #1}\hskip \labelsep {\bfseries #2.}]}{\end{trivlist}}
\newenvironment{hypothesis}[2][Hipoteza]{\begin{trivlist}
\item[\hskip \labelsep {\bfseries #1}\hskip \labelsep {\bfseries #2.}]}{\end{trivlist}}
\newenvironment{lemma}[2][Lemat]{\begin{trivlist}
\item[\hskip \labelsep {\bfseries #1}\hskip \labelsep {\bfseries #2.}]}{\end{trivlist}}
\newenvironment{exercise}[2][Ćwiczenie]{\begin{trivlist}
\item[\hskip \labelsep {\bfseries #1}\hskip \labelsep {\bfseries #2.}]}{\end{trivlist}}
\newenvironment{reflection}[2][Uwaga]{\begin{trivlist}
\item[\hskip \labelsep {\bfseries #1}\hskip \labelsep {\bfseries #2.}]}{\end{trivlist}}
\newenvironment{proposition}[2][Założenie]{\begin{trivlist}
\item[\hskip \labelsep {\bfseries #1}\hskip \labelsep {\bfseries #2.}]}{\end{trivlist}}
\newenvironment{corollary}[2][Wniosek]{\begin{trivlist}
\item[\hskip \labelsep {\bfseries #1}\hskip \labelsep {\bfseries #2.}]}{\end{trivlist}}

\begin{document}

\title{Aproksymacja}
\author{Grams, Stanisław\\Jezierski, Maciej\\Korczakowski, Juliusz\\ MFI UG\\Algorytmy Numeryczne}

\maketitle
\section {Operacje na macierzach}
\subsection{O implementacji}
Program \textit{„approximations”} został napisany w języku C++, a wyniki działania programu zapisywane są do poszczególnych plików \textit{*.csv}.
\subsection{Zaimplementowane algorytmy}
\begin{itemize}
	\item Algorytm Gaussa z częściowym wyborem elementu
	\item Algorytm Gaussa z optymalizacją dla macierzy rzadkich
	\item Algorytm Gaussa-Seidela
	\item Algorytm Gaussa-Seidela z optymalizacją dla macierzy rzadkich
\\
\\
Użyto także metody SparseLU z biblioteki Eigen3.
\end{itemize}

\section{Wnioski}

\subsection{Błąd aproksymacji}
\centering
	\begin{tabular}{|C{2.5cm}| C{5cm}|C{5cm}|}
		\hline
		\textbf{} & \textbf{Generowanie} & \textbf{Obliczanie} \\ 
		
		\hline
		Gauss & 0 & 0  \\ 
		
		\hline
		Gauss Sparse & 0 & 0 \\ 
		 
		 \hline
		Gauss Seidel & 0 & 0\\ 
		
		\hline
		SpraseLU & 0 & 0\\
		
		\hline
	\end{tabular}


\section{Podział pracy}
\centering
	\begin{tabular}{| p{5cm} | p{5cm} | p{5cm} |}
		\hline
		\textbf{Stanisław Grams} & \textbf{Juliusz Korczakowski} & \textbf{Maciej Jezierski} \\ \hline
		Implementacja algorytmu Gaussa-Seidela & Implementacja algorytmu Jacobiego & Implementacja algorytmu PG oraz PGS  \\ \hline
		 Implementacja symulacji Monte Carlo& Przygotowanie testów i ich uruchomienie &Analiza wykresów oraz przygotowanie sprawozdania \\ \hline
		Implementacja algorytmu generowania macierzy & Przygotowanie wykresów końcowych &Praca nad strukturą projektu\\ \hline
	\end{tabular}
\end{document}

\]
\end{document}