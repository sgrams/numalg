\documentclass[10pt]{article}
 
\usepackage[margin=1.5cm]{geometry} 
\usepackage{amsmath,amsthm,amssymb}
\usepackage{polski}
\usepackage[utf8]{inputenc}
\usepackage{siunitx}
\usepackage{graphicx}
\usepackage{comment}


 
\newenvironment{theorem}[2][Twierdzenie]{\begin{trivlist}
\item[\hskip \labelsep {\bfseries #1}\hskip \labelsep {\bfseries #2.}]}{\end{trivlist}}
\newenvironment{question}[2][Pytanie]{\begin{trivlist}
\item[\hskip \labelsep {\bfseries #1}\hskip \labelsep {\bfseries #2.}]}{\end{trivlist}}
\newenvironment{hypothesis}[2][Hipoteza]{\begin{trivlist}
\item[\hskip \labelsep {\bfseries #1}\hskip \labelsep {\bfseries #2.}]}{\end{trivlist}}
\newenvironment{lemma}[2][Lemat]{\begin{trivlist}
\item[\hskip \labelsep {\bfseries #1}\hskip \labelsep {\bfseries #2.}]}{\end{trivlist}}
\newenvironment{exercise}[2][Ćwiczenie]{\begin{trivlist}
\item[\hskip \labelsep {\bfseries #1}\hskip \labelsep {\bfseries #2.}]}{\end{trivlist}}
\newenvironment{reflection}[2][Uwaga]{\begin{trivlist}
\item[\hskip \labelsep {\bfseries #1}\hskip \labelsep {\bfseries #2.}]}{\end{trivlist}}
\newenvironment{proposition}[2][Założenie]{\begin{trivlist}
\item[\hskip \labelsep {\bfseries #1}\hskip \labelsep {\bfseries #2.}]}{\end{trivlist}}
\newenvironment{corollary}[2][Wniosek]{\begin{trivlist}
\item[\hskip \labelsep {\bfseries #1}\hskip \labelsep {\bfseries #2.}]}{\end{trivlist}}
 
\begin{document}

\title{Implementacja protokołów populacyjnych\\oraz testy wydajnościowe}
\author{Grams, Stanisław\\Jezierski, Maciej\\Korczakowski, Juliusz\\ MFI UG\\Algorytmy Numeryczne}

\maketitle
\section {Operacje na macierzach}
\subsection{O implementacji}
Program \textit{„gauss”} został napisany w języku C++ z użyciem bibliotek z standardu C++. 
Wyniki działania programu zapisywane są do poszczególnych plików \textit{*.csv}.
\subsection{Zaimplementowane algorytmy}
\begin{itemize}
	\item (PG - Partial Gauss) Algorytm Gaussa z częściowym wyborem elementu
	\Item (PGO - Partial Gauss Optimised) Algrotym Gaussa z optymalizacją dla macierzy rzadkich
	\item Algorytm Jacobiego z postacią iteracyjną:
	\item Algorytm Gaussa-Seidela
	\item Algorytm Jacobiego
\end{itemize}

\section{Majority/Consensus problem}
W poniższym tekście przedstawimy analizę problemu przeprowadzenia głosowania większościowego. Jego przebieg wygląda następująco:\\
Dana jest sprecyzowana, naturalna wartość reprezentująca ilość agentów, znajdujących się w 3 możliwych stanach:\\
\begin{itemize}
    \item Yes (Y) - agent głosujący na TAK
    \item No (N) - agent głosujący na NIE
    \item Unknown (U) - agent niezdecydowany
\end{itemize}

W każdym kroku losowanych jest dwóch agentów z równomiernym prawdopodobieństwem, których stan zostaje zmieniony według wcześniej wytyczonych reguł wyglądających następująco:
\begin{itemize}
	\item $ Y,U \to Y,Y $
	\item $ Y,N \to U,U $
	\item $ N,U \to N,N $
\end{itemize}
Powyższe zamiany dokonywane są w pętli aż do uzyskania jednakowego stanu u wszystkich agentów.\\
\\
Program przygotowany w ramach projektu, ma na celu obliczenie prawdopodobieństwa zagłosowania na TAK przy danej liczbie agentów i określonym stanie początkowym. Prawdopodobieństwo obliczane jest za pomocą algorytmów wymienionych w sekcji 1.2.

\section{Implementacja i możliwość stosowania metod iteracyjnych}
\begin{figure}[h]
	\caption{Wykres reprezentujący błąd bezwzględny metod Gaussa oraz metod iteracyjnych względem metody Monte Carlo\label{rys}}
	\centering
	\begin{subfigure}{0.5\textwidth}
		\includegraphics[width=\textwidth]{1.png}
		\caption{ \label{Rys1a}}
	\end{subfigure}
\end{figure}
\subsection{Generowanie układu równań dla danej liczby agentów}
Generowanie układu równań dla danego $N$ odbywa się w sposób następujący:
\begin{enumerate}
	\item Określenie wszystkich możliwych przypadków (ilość agentów $\#Y$ oraz ilość agentów $\#N$),
	\item Wyliczenie wszystkich możliwych kombinacji bez powtórzeń za pomocą Symbolu Newtona ${{N} \choose {2}}$,
	\item Wygenerowanie równań dla poszczególnych przypadków,
	\item Osadzenie równań w macierzy,
	\item Wypełnienie wektora $B$ zerami z wyjątkiem ostatniej wartości (gdyż ostatni przypadek jest zawsze przypadkiem pewnym, tj $P_{\#Y=N,\#N=0}=1)$.
\end{enumerate}
\subsection{Prawidłowość implementacji}
By zweryfikować poprawność implementacji zarówno generowania macierzy jak i obliczania stworzonego w ten sposób układu równań, wykonane zostały testy dla $N = 3,4,...,15$, których zadaniem było obliczenie wszystkich możliwych prawodopodobieństw i zestawienie ich z prawdopodobieństwem wyliczonym za pomocą metody Monte Carlo w ilości iteracji $=1000000$.
Błędy osiągnięte za pomocą zaimplementowanych algorytmów osiągają wartości rzędu $[10^{-3},10^{-6}]$ (przy czym warto zauważyć, że tak wysoki błąd osiągają tylko metody iteracyjne z niską żądaną precyzją), co biorąc pod uwagę niedoskonałość metody Monte Carlo jest wynikiem jak najbardziej zadowalającym.
\subsection{Metody iteracyjne a problem}
Wnioskując z wykresu \ref{Rys1a} możemy śmiało stwierdzić, iż metody iteracyjne są jak najbardziej słusznym sposobem na rozwiązanie problemu. Jednakże, najistotniejszym czynnikiem w przypadku ich działania jest zakładana dokładność obliczeń, tj. $\|{X^{(k+1)}-X^{(k)}}\| < p$, gdzie $p$ - żądana precyzja.
W przypadku żądanej dokładności równej $10^{-6}$ zauważyć można że, różnice względem wartości wyliczonej za pomocą metody Monte Carlo są większe niż w przypadku metod iteracyjnych z większą zadaną dokładnością ($10^{-10}, 10^{-14}$).
\begin{wn}
	Metody iteracyjne umożliwiają rozwiązanie problemu aczkolwiek, by osiągnąć dokładniejsze wyniki, należy zwiększyć dokładność, a co za tym idzie - liczbę iteracji, co znacząco wydłuża czas działania algorytmu. \label{wn:1}
\end{wn}
\section{Analiza wyników i wydajność zaimplementowanych algorytmów}
\begin{figure}[h]
	\caption{Wykresy reprezentujące czas wykonania i błędy bezwzględne zaimplementowanych algorytmów \label{rys}}
	\begin{subfigure}{0.5\textwidth}
		\includegraphics[width=\textwidth]{2.png}
		\caption{ \label{Rys2a}}
	\end{subfigure}
	\begin{subfigure}{0.5\textwidth}
	\includegraphics[width=\textwidth]{3.png}
	\caption{ \label{Rys2b}}
	\end{subfigure}
	\begin{subfigure}{0.5\textwidth}
	\includegraphics[width=\textwidth]{4.png}
	\caption{ \label{Rys2c}}
\end{subfigure}
	\begin{subfigure}{0.5\textwidth}
	\includegraphics[width=\textwidth]{5.png}
	\caption{ \label{Rys2d}}
\end{subfigure}
	\begin{subfigure}{0.5\textwidth}
	\includegraphics[width=\textwidth]{6.png}
	\caption{ \label{Rys2e}}
\end{subfigure}
\end{figure}
\subsection{Analiza wyników}
Błąd bezwzględny jest wyliczany za pomocą  ??

\subsubsection{Gauss oraz Gauss z optymalizacją dla macierzy rzadkich}
Przeanalizujmy wykres na, którym zostały zestawione wyniki dla błędu algorytmu PG or PGO. Z wykresu jednoznacznie wynika, że błąd był identyczny, więc powyższa optymalizacja nie ma wpływu na wynik.

\subsubsection{Algorytmy iteracyjne}
Zarówno algorytm Jacobiego jak i Gaussa-Seidela oferują taką samą dokładność, w zależności od tego jaka precyzja była żądana.
\subsection{Wydajność}

\subsubsection{Metody PG oraz PGS}
Przeanalizujmy wykres \ref{Rys2b}. Zauważyć na nim można znaczną przewagę algorytmu PGS względem PG w czasie wykonywania. Wynika to ze specyfiki działania wariantu PGS - algorytm ten pomija redukcję elementów w wierszu w przypadku gdy wybrany na początku element jest równy zeru.
\begin{wniosek}
	Wariant PGS algorytmu Gaussa jest wydajniejszy niż standardowy wariant PG. Zestawiając ten wniosek z wnioskiem \ref{wn:3} stwierdzić można, iż wariant PGS zapewnia o wiele lepszą wydajność nie mając żadnego wpływu na poprawność zwracanych wyników.
\end{wniosek}
\subsubsection{Metody iteracyjne}
W przypadku metod iteracyjnych, należy rozważyć osobno algorytmy dla różnej żądanej precyzji.
Jednakże we wszystkich przypadków, zależność jest następująca: Metoda Gaussa-Seidela w każdym przypadku (wraz z wzrostem $N$) ma krótszy czas wykonania względem metody Jacobiana.
\begin{wniosek}
	Metoda Gaussa-Seidela jest wydajniejsza od metody Jacobiana - wynika to ze sposobu działania obu tych algorytmów. Metoda Jacobiana by osiągnąć żądaną precyzję musi wykonać o wiele więcej iteracji niż metoda Gaussa-Seidela.
\end{wniosek}
\subsection{Podsumowanie}
\begin{wniosek}
	Analizując wyniki pod kątem wydajności i dokładności metodą o najlepszym stosunku tych dwóch czynników okazała się PGO. Jeżeli jednak ta metoda oferuje zbyt małą dokładność należałoby użyć metody Gaussa-Seidela.
\end{wniosek}
\section{Podział pracy}
\centering
	\begin{tabular}{| p{5cm} | p{5cm} | p{5cm} |}
		\hline
		\textbf{Stanisław Grams} & \textbf{Juliusz Korczakowski} & \textbf{Maciej Jezierski} \\ \hline
		Implementacja algorytmu PG oraz PGS & Implementacja algorytmu Gaussa-Seidela & Implementacja algorytmu Jacobiego \\ \hline
		Praca nad strukturą projektu & Przygotowanie testów i ich uruchomienie & Przygotowanie sprawozdania \\ \hline
		Implementacja algorytmu generowania macierzy & Przygotowanie wykresów końcowych &Implementacja symulacji Monte Carlo\\ \hline
		
		
		
	\end{tabular}
\end{document}
