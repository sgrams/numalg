\documentclass[12pt]{article}
 
\usepackage[margin=1in]{geometry} 
\usepackage{amsmath,amsthm,amssymb}
\usepackage{polski}
\usepackage[utf8]{inputenc}
\usepackage{siunitx}
 
\newenvironment{theorem}[2][Twierdzenie]{\begin{trivlist}
\item[\hskip \labelsep {\bfseries #1}\hskip \labelsep {\bfseries #2.}]}{\end{trivlist}}
\newenvironment{question}[2][Pytanie]{\begin{trivlist}
\item[\hskip \labelsep {\bfseries #1}\hskip \labelsep {\bfseries #2.}]}{\end{trivlist}}
\newenvironment{hypothesis}[2][Hipoteza]{\begin{trivlist}
\item[\hskip \labelsep {\bfseries #1}\hskip \labelsep {\bfseries #2.}]}{\end{trivlist}}
\newenvironment{lemma}[2][Lemat]{\begin{trivlist}
\item[\hskip \labelsep {\bfseries #1}\hskip \labelsep {\bfseries #2.}]}{\end{trivlist}}
\newenvironment{exercise}[2][Ćwiczenie]{\begin{trivlist}
\item[\hskip \labelsep {\bfseries #1}\hskip \labelsep {\bfseries #2.}]}{\end{trivlist}}
\newenvironment{reflection}[2][Uwaga]{\begin{trivlist}
\item[\hskip \labelsep {\bfseries #1}\hskip \labelsep {\bfseries #2.}]}{\end{trivlist}}
\newenvironment{proposition}[2][Założenie]{\begin{trivlist}
\item[\hskip \labelsep {\bfseries #1}\hskip \labelsep {\bfseries #2.}]}{\end{trivlist}}
\newenvironment{corollary}[2][Wniosek]{\begin{trivlist}
\item[\hskip \labelsep {\bfseries #1}\hskip \labelsep {\bfseries #2.}]}{\end{trivlist}}
 
\begin{document}

\title{Implementacja rozwinięć wybranych funkcji\\w szereg Maclaurina}
\author{Grams, Stanisław\\251000 MFI UG\\Algorytmy Numeryczne}

\maketitle

\section{Sumowanie szeregów potęgowych}

\subsection{Rozwinięcia funkcji w szereg Maclaurina}
$$ \forall{x} \sin x = \sum_{n=0}^{\infty} \frac{(-1)^n}{(2n+1)!} x^{2n+1} = x - \frac{x^3}{3!} + \frac{x^5}{5!} - ...$$
$$ \forall {x} \exp{x} = \sum_{n=0}^{\infty} \frac{x^n}{n!} = 1 + x + \frac{x^2}{2!} + \frac{x^3}{3!} + ...$$

\subsection{O implementacji}
Program \textit{„taylor”} został napisany w języku C z użyciem bibliotek:
\begin{itemize}
    \item \textit{„quadmath.h”} — pozwalającej na użycie zmiennych typu $\_\_float128$,
    \item \textit{„pthread.h”} — do celów obsługi wielowątkowości,
    \item \textit{„glib.h”} — ułatwiającej pracę z tablicami znaków.
\end{itemize}
Wyniki działania programu zapisywane są do poszczególnych plików \textit{*.csv}.

\subsection{Wyniki}
Program uruchamiano na przedziale $x \in [-10; 9.99999]$ dla kolejnych parametrów\\$M = 4, 8, 16, 32, 64$ z krokiem $0.00001$ (\num{2e6} elementów). Z powodu braku wspomagania koprocesora zmiennoprzecinkowego dla typu poczwórnej precyzji, obliczenia zajmowały bardzo dużo czasu (dla $M = 256$ nawet \textbf{16} godzin!).

\subsection{Konkluzje po implementacji i przeprowadzeniu doświadczeń}
\subsubsection{Odpowiedzi na postawione hipotezy}

\begin{hypothesis}{1}
Sumowanie od końca daje dokładniejsze wyniki niż sumowanie od początku.
\end{hypothesis}
Hipoteza dla zaimplementowanych funkcji, na przedziale $x \in [-10; 9.99999]$, jest prawdziwa \textbf{jedynie} w pierwszym przypadku, tj. sumując elementy szeregu potęgowego obliczane bezpośrednio ze wzoru Taylora w kolejności od końca.\\
Dowód: patrz \textit{Hipoteza 3}.

\begin{hypothesis}{2}
Używając rozwinięcia wokół 0 (szereg Maclaurina), przy tej samej liczbie składników szeregu dokładniejsze wyniki uzyskujemy przy małych argumentach.
\end{hypothesis}
Zaimplementowane funkcje potwierdzają hipotezę na przedziale $x \in [-10; 9.99999]$, co możemy łatwo zauważyć na wykresie \textit{„01\_functions\_m002.svg”}. Funkcje 1, 2, 3 oraz 4 zbiegają się z wykresem funkcji 0 tylko w okolicach zera oraz w całkowitych wielokrotnościach liczby $2\pi$ (ze względu na użyte sprowadzenie do przedziału $[-2\pi;2\pi]$).

\begin{hypothesis}{3}
Sumowanie elementów obliczanych na podstawie poprzedniego daje dokładniejsze wyniki niż obliczanych bezpośrednio ze wzoru.
\end{hypothesis}
Zaimplementowane funkcje pokazują, że na przedziale $x \in [-10;9.99999]$ hipoteza jest \\prawdziwa, co więcej, możemy powiedzieć, że w przypadku użycia typu $\_\_float128$, funkcje te mają taką samą dokładność, jak funkcja 2 (szereg Maclaurina, sumowanie od końca).\\
Dowodem niech będą wyniki średnich arytmetycznych różnic w stosunku do obliczeń popełnionych funkcjami wbudowanymi:\\
\noindent\texttt{func1() avg. absolute diff to func0() with M=16 is}\\
\textsc{\textbf{0.000000000001653952179636059798590}7078303108796753612877507503184}\\
\texttt{func2() avg. absolute diff to func0() with M=16 is}\\
\textsc{\textbf{0.000000000001653952179636059798590}828005598135744063693864285095}1\\
\texttt{func3() avg. absolute diff to func0() with M=16 is}\\
\textsc{\textbf{0.000000000001653952179636059798590}8280055981357440636938642850951}\\
\texttt{func4() avg. absolute diff to func0() with M=16 is}\\
\textsc{\textbf{0.000000000001653952179636059798590}8280055981357440636938642850951}\\
Na powyższych wynikach, dla parametru $M = 16$, zauważyć możemy dokładność co do 33 miejsc po przecinku.

\subsubsection{Odpowiedzi na postawione pytania}
\begin{question}{1}
Jak zależy dokładność obliczeń (błąd) od liczby sumowanych składników?
\end{question}
Im wyższa liczba sumowanych składników (parametr M), tym większa dokładność obliczeń (niższy błąd bezwzględny do obliczeń wykonanych przez funkcje wbudowane), oraz tym dłuższy czas wykonania programu.

\subsection{Spostrzeżenia}
\begin{enumerate}
    \item Użycie \_\_float128 – typu nie wspieranego przez wbudowany koprocesor znacznie zwiększa czas działania programu.
    \item Spośród zbadanych parametrów wystarczająca dokładność uzyskiwana jest już od $M = 32$.
    \item Sprowadzanie argumentu do przedziału $[-2\pi;2\pi]$ zwiększa dokładność obliczeń, przyspiesza ich czas oraz zwiększa stabilność programu.
    \item Znaczący przyrost wydajności można uzyskać przez podzielenie obliczeń pomiędzy kilka wątków.
\end{enumerate}

\subsection{Wykresy}
Wykresy (w postaci wektorowej) załączono w katalogu \textit{„report/plots/”}, natomiast wyniki średnich arytmetycznych w pliku \textit{„report/diffs.txt”}\\
Do narysowania wykresów użyto oprogramowania \textit{„GNU Octave”} wraz z pakietem \textit{„gnuplot”}. Skrypty załączono w katalogu \textit{„scripts/”}.

\end{document}
