\documentclass[12pt]{article}
 
\usepackage[margin=1in]{geometry} 
\usepackage{amsmath,amsthm,amssymb}
\usepackage{polski}
\usepackage[utf8]{inputenc}
\usepackage{siunitx}
 
\newenvironment{theorem}[2][Twierdzenie]{\begin{trivlist}
\item[\hskip \labelsep {\bfseries #1}\hskip \labelsep {\bfseries #2.}]}{\end{trivlist}}
\newenvironment{question}[2][Pytanie]{\begin{trivlist}
\item[\hskip \labelsep {\bfseries #1}\hskip \labelsep {\bfseries #2.}]}{\end{trivlist}}
\newenvironment{hypothesis}[2][Hipoteza]{\begin{trivlist}
\item[\hskip \labelsep {\bfseries #1}\hskip \labelsep {\bfseries #2.}]}{\end{trivlist}}
\newenvironment{lemma}[2][Lemat]{\begin{trivlist}
\item[\hskip \labelsep {\bfseries #1}\hskip \labelsep {\bfseries #2.}]}{\end{trivlist}}
\newenvironment{exercise}[2][Ćwiczenie]{\begin{trivlist}
\item[\hskip \labelsep {\bfseries #1}\hskip \labelsep {\bfseries #2.}]}{\end{trivlist}}
\newenvironment{reflection}[2][Uwaga]{\begin{trivlist}
\item[\hskip \labelsep {\bfseries #1}\hskip \labelsep {\bfseries #2.}]}{\end{trivlist}}
\newenvironment{proposition}[2][Założenie]{\begin{trivlist}
\item[\hskip \labelsep {\bfseries #1}\hskip \labelsep {\bfseries #2.}]}{\end{trivlist}}
\newenvironment{corollary}[2][Wniosek]{\begin{trivlist}
\item[\hskip \labelsep {\bfseries #1}\hskip \labelsep {\bfseries #2.}]}{\end{trivlist}}
 
\begin{document}

\title{Implementacja algorytmu eliminacji Gaussa\\oraz testy wydajnościowe i obliczeniowe trzech określonych typów}
\author{Grams, Stanisław\\Jezierski, Maciej\\Korczakowski, Juliusz\\ MFI UG\\Algorytmy Numeryczne}

\maketitle
\section {Operacje na macierzach}
\subsection{O sprawozdaniu}
Sprawozdanie prezentuje analizę wydajności i poprawności implementacji algorytmu eliminacji Gaussa, dla macierzy kwadratowej $A$ oraz wektora $B$ w układzie liniowym $A*X = B$ gdzie współczynniki wektora $A$ oraz macierzy kwadratowej $X$ są losowane. \\
Zakres losowanych współczynników jest równy $[-1, 1]$, natomiast każdy z nich został wyznaczony według wzoru $r/2^{16}$, gdzie $r$ jest losową liczbą zakresu $[-2^{16} , 2^{16}]$.
Wygenerowany w ten sposób wektor $X$ pozostaje próbą kontrolną natomiast do algorytmu jako parametr przekazujemy wyliczony według powyższego wzoru wektor $B$.
\subsection{O implementacji}
Program \textit{„gauss”} został napisany w języku C++ z użyciem bibliotek:
\begin{itemize}
    \item \textit{„gmp.h”} — pozwalającej zaimplementować wymagany typ całkowity (TC)
    \item \textit{„pthread.h”} — do celów obsługi wielowątkowości,
\end{itemize}
Wyniki działania programu zapisywane są do poszczególnych plików \textit{*.csv}.
Zaimplementowano następujące warianty algorytmu Gaussa dla kolejnych typów zmiennych rzeczywistych:
\begin{itemize}
\item[] Warianty algorytmu eliminacji
    \item G: bez wyboru elementu podstawowego,
    \item PG: z częściowym wyborem elementu podstawowego, 
    \item FG: z całkowitym wyborem elementu podstawowego.
\item[] Typy zmiennych rzeczywistych
    \item {float} – wbudowany typ pojedynczej precyzji 
    \item {double} – wbudowany typ podwójnej precyzji
    \item {MyType} – typ zaimplementowany przez zespół, oparty o bibliotekę „GMP.h” i typ „mpq\_t”
\end{itemize}
\subsection{Konkluzje}
\begin{hypothesis}{1}
Dla dowolnego ustalone rozmiaru macierzy czas działania metody Gaussa w kolejnych wersjach (G,PG,FG) rośnie.
\end{hypothesis}

\subsection{Wykresy}
Wykresy (w postaci wektorowej) załączono w katalogu \textit{„report/plots/”}\\
Do narysowania wykresów użyto oprogramowania \textit{„GNU Octave”} wraz z pakietem \textit{„gnuplot”}. Skrypty załączono w katalogu \textit{„scripts/”}.

\end{document}
